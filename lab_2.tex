\documentclass[leqno, 11pt]{article}

\usepackage{lmodern}
\usepackage[scaled]{beramono}
\usepackage[T1]{fontenc}
\usepackage{amssymb}
\usepackage{amsmath}
\usepackage{hyperref}
\hypersetup{%
  colorlinks=true,
  linkcolor=magenta,
  filecolor=magenta,
  urlcolor=magenta
}

\usepackage[margin=1in]{geometry}
\usepackage{listings}
\usepackage{graphicx}
\usepackage{caption}

\captionsetup{%
  width=1.0\linewidth,
  justification=centering
}

% Better `@' symbol
\newcommand{\at}{\mbox{}{\fontfamily{ptm}\selectfont @}\mbox{}}

\newcommand\blfootnote[1]{%
  \begingroup
    \renewcommand\thefootnote{}\footnote{#1}
    \addtocounter{footnote}{-1}
  \endgroup
}

%\graphicspath{"/home/brandon/Desktop/IFT_194/labs/photos"}

\usepackage{xcolor}
\definecolor{javacommentscolor}{HTML}{646464}
\definecolor{javakeywordscolor}{HTML}{7F0055}
\definecolor{javastringscolor}{HTML}{2A00FF}

\lstset{%
  basicstyle=\scriptsize\ttfamily, % code to be displayed as monospace
  breaklines=true,
  %frame=b
  commentstyle=\color{javacommentscolor},
  keywordstyle=\color{javakeywordscolor},
  stringstyle=\color{javastringscolor},
  showstringspaces=false,  % do not show string spaces character
  tabsize=4,  % change tabs to spaces
  keywordsprefix={@},  % capture method annotations and doctools
  %showtabs=true,
  %tab=|
}

\newcommand{\centeredimage}[2]{%
  \begin{center}
    \includegraphics[scale=#1]{#2}
  \end{center}
}

\title{\vspace{6ex}Fundamental Programming Structures in Java\\
  \Large IFT 194: Lab 2}
\author{Brandon Doyle\\
\href{mailto:bdoyle@asu.edu}{bdoyle5\at{}asu.edu}\\
1215232174\\[1em]
Dr. Usha Jagannathan\\
\href{mailto:Usha.Jagannathan@asu.edu}{Usha.Jagannathan\at{}asu.edu}}

\setlength{\parindent}{0em}
\setlength{\parskip}{0.5em}

\begin{document}
\begin{titlepage}
\clearpage\maketitle
\thispagestyle{empty}
\end{titlepage}
\section*{Pre-Lab Exercises}
\subsection*{A. Textbook Sections 5.1--5.3}
\begin{enumerate}
  \item We are tasked with rewriting various conditions in valid Java syntax. 
        \begin{enumerate}
          \item The condition \texttt{x > y > z} may be written in Java as \texttt{x > y \&\& y > z}, i.e. we need to join the two comparisons by the $\wedge$--logical operator. This is a result of the type of objects the relational operators act upon; because \texttt{x > y} returns a \texttt{boolean} type, we receive a compile-time error (invalid types).
    
                Interestingly enough, this \textit{is} valid Python syntax due its recursive \href{https://github.com/python/cpython/blob/master/Grammar/Grammar#L93}{comp\_op} Grammar definition, so we may (hypothetically) write an inifinite sequence \texttt{expr comp\_op ... expr comp\_op expr}. $\wedge$--logical operators are automatically inserted.
              \item The statement ``x and y are both less than 0'' may quite simply be expressed as \texttt{x < 0 \&\& y < 0}.
              \item The statement ``neither x nor y are less than 0'' may be expressed as \texttt{x >= 0 \&\& y >= 0}, or the negation of the previous predicate, i.e. \texttt{!(x < 0 \&\& y < 0)}. I think the former is more readable, however.
              \item The statement ``x equals y but not z'' may be written as \texttt{x == y \&\& x != z}.
        \end{enumerate}
  \item 
\end{enumerate}
\section*{Conclusion}
\blfootnote{View the source of this document on \href{https://github.com/bjd2385/IFT_194_labs/blob/master/\jobname}{GitHub}.}
\end{document}
