\documentclass[leqno, 11pt]{article}

\usepackage{lmodern}
\usepackage[scaled]{beramono}
\usepackage[T1]{fontenc}
\usepackage{amssymb}
\usepackage{tikz-uml}
\usepackage{amsmath}
\usepackage{hyperref}
\hypersetup{%
  colorlinks=true,
  linkcolor=magenta,
  filecolor=magenta,
  urlcolor=magenta
}

\usepackage[margin=1in]{geometry}
\usepackage{listings}
\usepackage{graphicx}
\usepackage{caption}

\captionsetup{%
  width=1.0\linewidth,
  justification=centering
}

\usepackage{enumitem} % allow characters instead of numbers
\usepackage{verbatimbox}  % center \verbatim env in figure env
\usepackage{siunitx}  % SI unit formatting
\usepackage{colortbl}  % color tabular rows/columns
\usepackage[most]{tcolorbox}

% TODO: Set up captions with tcolorbox so the user doesn't have to explicitly
%       pass a caption={} to each \lstlisting command.
\newtcolorbox[blend into=figures]{codefigure}[1][]{%
  enhanced jigsaw, 
  float=!ht, 
  breakable, 
  frame hidden, 
  colback=white,
  #1
}

% Better `@' symbol
\newcommand{\at}{\mbox{}{\fontfamily{ptm}\selectfont @}\mbox{}}

\newcommand\blfootnote[1]{%
  \begingroup
    \renewcommand\thefootnote{}\footnote{#1}
    \addtocounter{footnote}{-1}
  \endgroup
}

%\graphicspath{"/home/brandon/Desktop/IFT_194/labs/photos"}

\usepackage{xcolor}

\definecolor{tableheaderrow}{HTML}{CED3DB}

\definecolor{javacommentscolor}{HTML}{646464}
\definecolor{javakeywordscolor}{HTML}{7F0055}
\definecolor{javastringscolor}{HTML}{2A00FF}

\definecolor{background}{HTML}{F2F7FF}

\lstset{%
  basicstyle=\linespread{0.8}\scriptsize\ttfamily, % code to be displayed as monospace
  breaklines=true,
  %frame=b
  commentstyle=\color{javacommentscolor},
  keywordstyle=\color{javakeywordscolor},
  stringstyle=\color{javastringscolor},
  showstringspaces=false,  % do not show string spaces character
  tabsize=4,  % change tabs to spaces
  keywordsprefix={@},  % capture method annotations and doctools
  %showtabs=true,
  %tab=|,
  xleftmargin=1cm,
  morekeywords={var}
}

% TODO: Modify listings in the next lab so that I don't have to wrap in figure
%       env anymore -- makes the following lines irrelevant
\renewcommand{\lstlistingname}{Figure}

\newcommand{\centeredimage}[2]{%
  \begin{center}
    \includegraphics[scale=#1]{#2}
  \end{center}
}

%% Make a multi-page code figure (only way I've figured out how to do this conveniently as of yet).
% Label, caption, and code path
\newcommand{\iftcodefigure}[3]{%
  \begin{codefigure}
    \label{#1}
    \addtocounter{figure}{-1}
    \lstinputlisting[language=java,
                     morekeywords={var},
                     backgroundcolor=\color{background},
                     framexleftmargin=10pt,
                     framextopmargin=6pt,
                     framexbottommargin=6pt,
                     frame=tb,
                     framerule=0pt, 
                     caption={#2}, captionpos=b]{#3}
  \end{codefigure}
}

\makeatletter
% Top-align float-only pages
% https://tex.stackexchange.com/a/28565/96382
\setlength{\@fptop}{0pt}
\setlength{\@fpbot}{0pt plus 1fil}

% Set up the table of contents so that titles are displayed
\renewcommand*\contentsname{Summary}
\setcounter{secnumdepth}{0}

% Share Figure and listings counter
\AtBeginDocument{%
  \let\c@figure\c@lstlisting
  \let\thefigure\thelstlisting
  \let\ftype@lstlisting\ftype@figure
}
\makeatother

\title{\vspace{6ex}Inheritance and Polymorphism in Java\\
  \Large IFT 194: Lab 5}
\author{Brandon Doyle\\
\href{mailto:bdoyle@asu.edu}{bdoyle5\at{}asu.edu}\\
1215232174\\[1em]
Dr. Usha Jagannathan\\
\href{mailto:Usha.Jagannathan@asu.edu}{Usha.Jagannathan\at{}asu.edu}}

\setlength{\parindent}{0em}
\setlength{\parskip}{0.5em}

\begin{document}
\begin{titlepage}
\clearpage\maketitle
\thispagestyle{empty}
\end{titlepage}
\tableofcontents
\blfootnote{View the source of this document on \href{https://github.com/bjd2385/IFT_194_labs/blob/master/\jobname.tex}{GitHub}.}
\newpage
\section{Overriding the \textit{equals} method}
In this question we're asked to write a class \texttt{Player} that holds information about an athlete. The class must have an \texttt{equals}-method that compares two instances of \texttt{Player}. Please see \hyperref[fig:one]{Figure 2} for my solution. The following is an example session.
\begin{verbbox}[\mbox{}\scriptsize]
Please enter a player name: Brandon
Please enter a team name: ASU
Please enter a jersey number: 55

Please enter a player name: Jasmine
Please enter a team name: ASU
Please enter a jersey number: 27

Name:      Brandon
Team Name: ASU
Jersey:    55

Name:      Jasmine
Team Name: ASU
Jersey:    27

Brandon == Jasmine: false
\end{verbbox}
\begin{figure}[h!]
  \centering
  \theverbbox
  \caption{Example output from \texttt{Player.java} in \hyperref[fig:one]{Figure 2}.}
  \label{fig:three}
\end{figure}
\iftcodefigure{fig:one}{Player.java}{%
  /home/brandon/eclipse-workspace/ift_194_labs/src/lab_5/Player.java}\\
Although this code seemingly works well, in practice I believe duplication may be prevented using metaclasses, much in the same way metaclasses may be used to create singletons.

This part of the lab is straightforward since we've already previously considered the difference between, for example, \texttt{"example1" == "example2"} and \texttt{"example1".equals("example2")}.

Also, I would again like to highlight the rather useful differences between creating \texttt{String} objects in Java with \texttt{new String("ex")} as opposed to the more brief \texttt{"ex"} (i.e., not encapsulating the instantation with an explicit call to \texttt{String}): the former always creates a new instance, whereas the latter makes an implicit call to \textit{intern} the sequence. In other words, if memory or space isn't so much of an issue, yet performance is, I might use the former instantiation over the latter to prevent interning, and vice-versa.
\section{Another type of employee}
In this question we're tasked with extending the class hierarchy included in \texttt{Staff.java} in \hyperref[fig:two]{Figure 4}. (I condensed the classes to a single file so I could see the hierarchy/outline.) See \hyperref[fig:four]{Figure 3} for an example.
\begin{verbbox}[\mbox{}\scriptsize]
Name:    Clint
Address: 1 Lomb Mem. Dr.
Phone:   585-5903
Current hours: 14
Total Sales: 456.62
$2,914.02

Name:    Marylin
Address: Mars
Phone:   inf
Current hours: 40
Total Sales: 950.0
$5,332.40
\end{verbbox}
\begin{figure}[h!]
  \centering
  \theverbbox
  \caption{Example output from \texttt{Staff.java} in \hyperref[fig:two]{Figure 4}.}
  \label{fig:four}
\end{figure}
\iftcodefigure{fig:two}{Staff.java.}{%
  /home/brandon/eclipse-workspace/ift_194_labs/src/lab_5/Staff.java}
\section{Conclusion}
It took about 4 hours to complete this lab. One thing I learned more about is  \texttt{abstract} classes. I've known of this feature in Python (cf. \href{https://docs.python.org/3/library/abc.html}{abc}), but I've never used the same concept in Java before. For example, see \hyperref[fig:five]{Figure 5} for a rewrite of \texttt{StaffMember} in Python.

I didn't really come across any issues with these concepts, i.e. inheritance and polymorphism, but that's probably because I've seen them in a few different languages.
\iftcodefigure{fig:five}{staff.py.}{%
  /home/brandon/eclipse-workspace/ift_194_labs/src/lab_5/staff.py}
\end{document}
