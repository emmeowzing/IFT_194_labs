\documentclass[leqno, 11pt]{article}

\usepackage{lmodern}
\usepackage[scaled]{beramono}
\usepackage[T1]{fontenc}
\usepackage{amssymb}
\usepackage{amsmath}
\usepackage{hyperref}
\hypersetup{%
  colorlinks=true,
  linkcolor=magenta,
  filecolor=magenta,
  urlcolor=magenta
}

\usepackage[margin=1in]{geometry}
\usepackage{listings}
\usepackage{graphicx}
\usepackage{caption}

\captionsetup{%
  width=1.0\linewidth,
  justification=centering
}

\usepackage{enumitem} % allow characters instead of numbers
\usepackage{verbatimbox}  % center \verbatim env in figure env
\usepackage{siunitx}  % SI unit formatting
\usepackage{colortbl}  % color tabular rows/columns
\usepackage[most]{tcolorbox}

% TODO: Set up captions with tcolorbox so the user doesn't have to explicitly
%       pass a caption={} to each \lstlisting command.
\newtcolorbox[blend into=figures]{codefigure}[1][]{%
  enhanced jigsaw, 
  float=!ht, 
  breakable, 
  frame hidden, 
  colback=white,
  #1
}

% Better `@' symbol
\newcommand{\at}{\mbox{}{\fontfamily{ptm}\selectfont @}\mbox{}}

\newcommand\blfootnote[1]{%
  \begingroup
    \renewcommand\thefootnote{}\footnote{#1}
    \addtocounter{footnote}{-1}
  \endgroup
}

%\graphicspath{"/home/brandon/Desktop/IFT_194/labs/photos"}

\usepackage{xcolor}

\definecolor{tableheaderrow}{HTML}{CED3DB}

\definecolor{javacommentscolor}{HTML}{646464}
\definecolor{javakeywordscolor}{HTML}{7F0055}
\definecolor{javastringscolor}{HTML}{2A00FF}

\lstset{%
  basicstyle=\linespread{0.8}\scriptsize\ttfamily, % code to be displayed as monospace
  breaklines=true,
  %frame=b
  commentstyle=\color{javacommentscolor},
  keywordstyle=\color{javakeywordscolor},
  stringstyle=\color{javastringscolor},
  showstringspaces=false,  % do not show string spaces character
  tabsize=4,  % change tabs to spaces
  keywordsprefix={@},  % capture method annotations and doctools
  %showtabs=true,
  %tab=|,
  xleftmargin=1cm % distinguish between text and code better
}

% TODO: Modify listings in the next lab so that I don't have to wrap in figure
%       env anymore -- makes the following lines irrelevant
\renewcommand{\lstlistingname}{Figure}

\newcommand{\centeredimage}[2]{%
  \begin{center}
    \includegraphics[scale=#1]{#2}
  \end{center}
}

% Label, caption, and code path
\newcommand{\iftcodefigure}[3]{%
  \begin{codefigure}
    \label{#1}
    \addtocounter{figure}{-1}
    \lstinputlisting[language=java, caption={#2}, captionpos=b]{%
      #3}
  \end{codefigure}
}

\makeatletter
% Top-align float-only pages
% https://tex.stackexchange.com/a/28565/96382
\setlength{\@fptop}{0pt}
\setlength{\@fpbot}{0pt plus 1fil}

% Share Figure and listings counter
\AtBeginDocument{%
  \let\c@figure\c@lstlisting
  \let\thefigure\thelstlisting
  \let\ftype@lstlisting\ftype@figure
}
\makeatother

\title{\vspace{6ex}Fundamental Programming Structures in Java\\
  \Large IFT 194: HW 2}
\author{Brandon Doyle\\
\href{mailto:bdoyle@asu.edu}{bdoyle5\at{}asu.edu}\\
1215232174\\[1em]
Dr. Usha Jagannathan\\
\href{mailto:Usha.Jagannathan@asu.edu}{Usha.Jagannathan\at{}asu.edu}}

\setlength{\parindent}{0em}
\setlength{\parskip}{0.5em}

\begin{document}
\begin{titlepage}
\clearpage\maketitle
\thispagestyle{empty}
\end{titlepage}
\blfootnote{View the source of this document on \href{https://github.com/bjd2385/IFT_194_labs/blob/master/\jobname.tex}{GitHub}.}
\section*{Section 2.1}
\subsection*{2.1}
The difference between the expressions \texttt{4}, \texttt{4.0}, \texttt{'4'}, and \texttt{"4"} is the type. For instance, the first is an \texttt{int}, the second may be a \texttt{double} or \texttt{float}, the third is a \texttt{char}, and the fourth is a \texttt{String} object.
\subsection*{2.3}
We next examine the output of the following code fragment.
\begin{lstlisting}[language=java, xleftmargin=0.3\textwidth]
System.out.print("Here we go!");
System.out.println("12345");
System.out.print("Test this if you are not sure.");
System.out.print("Another.");
System.out.println();
System.out.println("All done.");
\end{lstlisting}
We expect this to output the following.
\begin{verbbox}
Here we go!12345
Test this if you are not sure.Another.
All done.
\end{verbbox}
\begin{center}
\theverbbox
\end{center}
This is purely a result of using \texttt{println} as opposed to just \texttt{print}. Moreover, \texttt{println} can be \href{https://docs.oracle.com/javase/10/docs/api/java/io/PrintStream.html#println()}{called without any arguments}.
\subsection*{2.4}
We're asked to determine what is wrong with the following statement.
\begin{verbbox}
System.out.println("To be or not to be, that is the question.");
\end{verbbox}
\begin{center}
\theverbbox
\end{center}
I honestly don't think there could be anything wrong with this statement, unless it's pointing at the fact that we may not continue strings on more than one line (as it may be formatted in the text). In that case, it is required that we use concatenation as
\begin{verbbox}
System.out.println("To be or not to be, that"
    + " is the question");
\end{verbbox}
\begin{center}
\theverbbox
\end{center}
\subsection*{2.8}
We are asked to determine which value is contained in the primitive \texttt{int} variable \texttt{depth} after the following lines execute.
\begin{verbbox}
size = 18;
size = size + 12;
size = size * 2;
size = size / 4;
\end{verbbox}
\begin{center}
\theverbbox
\end{center}
These statements evaluate to
\begin{align*}
\dfrac{2\left(18 + 12\right)}{4}&=\boxed{15.}
\end{align*}
\subsection*{2.11}
\begin{lstlisting}[language=java, xleftmargin=0.25\textwidth]
int iResult, num1 = 25, num2 = 40, num3 = 17, num4 = 5;
double fResult, val1 = 17.0, val2 = 12.78;
\end{lstlisting}
Given the above declarations, we're tasked with determining the result of several statements.
\begin{enumerate}[label=\alph*.]
  \itemsep-0.2em
  % a
  \item \texttt{iResult = num1 / num4} will store the integer 5 in \texttt{iResult}.
  % b
  \item \texttt{fResult = num1 / num4} will store the double 5.0 in \texttt{fResult}.
  % c
  \item \texttt{iResult = num3 / num4} will store the integer 3 in \texttt{iResult}.
  % d
  \item \texttt{fResult = num3 / num4} will store the double 3.0 in \texttt{fResult}.
  % e
  \item \texttt{fResult = val1 / num4} will store the double 3.4 in \texttt{fResult}.
  % f
  \item \texttt{fResult = val1 / val2} will store the double 1.33... in \texttt{fResult}.
  % g
  \item \texttt{iResult = num1 / num2} will store the integer 0 in \texttt{iResult}.
  % h
  \item \texttt{fResult = (double)num1 / num2} will store the double 0.625 in \texttt{fResult}.
  % i
  \item \texttt{fResult = num1 / (double)num2} will store the double 0.625 in \texttt{fResult}.
  % j
  \item \texttt{fResult = (double)(num1 / num2)} will store the double 0.0 in \texttt{fResult}.
  % k
  \item \texttt{iResult = (int)(val1 / num4)} will store the integer \texttt{3} in \texttt{iResult}.
  % l
  \item \texttt{fResult = (int)(val1 / num4)} will store the double \texttt{3.0} in \texttt{fResult}.
  % m
  \item \texttt{fResult = (int)((double)num1 / num2)} will store the double 0.0 in \texttt{fResult}.
  % n
  \item \texttt{iResult = num3 \% num4} will store the integer 2 in \texttt{iResult}.
  % o
  \item \texttt{iResult = num2 \% num3} will store the integer 6 in \texttt{iResult}.
  % p
  \item \texttt{iResult = num3 \% num2} will store the integer 17 in \texttt{iResult}.
  % q
  \item \texttt{iResult = num2 \% num4} will store the integer 0 in \texttt{iResult}.
\end{enumerate}
\subsection*{2.12}
We're tasked with stating the order in which the following expressions will be evaluated by providing a number beneath each operator.
\begin{enumerate}[label=\alph*.]
  \itemsep-1em
  % a
  \item \begin{verbbox}
a - b - c - d
  1   2   3
        \end{verbbox}
        \theverbbox
  % b
  \item \begin{verbbox}
a - b + c - d
  1   2   3
        \end{verbbox}
        \theverbbox
  % c
  \item \begin{verbbox}
a + b / c / d
  3   1   2
        \end{verbbox}
        \theverbbox
  % d
  \item \begin{verbbox}
a + b / c * d
  3   1   2
        \end{verbbox}
        \theverbbox
  % e
  \item \begin{verbbox}
a / b * c * d
  1   2   3
        \end{verbbox}
        \theverbbox
  % f
  \item \begin{verbbox}
a % b / c * d
  1   2   3
        \end{verbbox}
        \theverbbox
  % g
  \item \begin{verbbox}
a % b % c % d
  1   2   3
        \end{verbbox}
        \theverbbox
  % h
  \item \begin{verbbox}
a - (b - c) - d
  2    1    3
        \end{verbbox}
        \theverbbox
  % i
  \item \begin{verbbox}
(a - (b - c)) - d
   2    1     3
        \end{verbbox}
        \theverbbox
  % j
  \item \begin{verbbox}
a - ((b - c) - d)
  3     1    2
        \end{verbbox}
        \theverbbox
  % k
  \item \begin{verbbox}
a % (b % c) * d * e
  2    1    3   4
        \end{verbbox}
        \theverbbox
  % l
  \item \begin{verbbox}
a + (b - c) * d - e
  3    1    2   4
        \end{verbbox}
        \theverbbox
  % m
  \item \begin{verbbox}
(a + b) * c + d * e
   1    2   4   3
        \end{verbbox}
        \theverbbox
\end{enumerate}
\section*{Section 2.2}
\subsection*{5.2}
We are tasked with determining what the issue may be with the \texttt{if-else} blocks in the following source.
\begin{figure}[h!]
  \centering
  \lstinputlisting[language=java]{%
    /home/brandon/eclipse-workspace/ift_194_hw/src/hw_2/Fragment1.java}
  \caption{Fragment1.java.}
  \label{fig:one}
\end{figure}\\
The problem is actually a \href{https://stackoverflow.com/a/2125207/3928184}{well-documented feature} of C-like languages: because C and Java ignore whitespace, indentation, despite being important for readability, is also ignored by the lexer. This means that Java has no way of knowing if we meant
\begin{verbbox}
if (X)
    if (Y)
        <block>
else
    <another block>
\end{verbbox}
\begin{center}
  \theverbbox
\end{center}
or
\begin{verbbox}
if (X)
    if (Y)
        <block>
    else
        <another block>
\end{verbbox}
\begin{center}
  \theverbbox
\end{center}
By default, Java appears to pair this \texttt{else}-statement with the nested \texttt{if}, which can be seen by setting \texttt{sum = 3} in my example in \autoref{fig:one} (which produces an incorrect result).
\subsection*{5.3}
We are asked if the following code segment is valid if part of an otherwise valid program.
\begin{figure}[h!]
  \centering
  \begin{lstlisting}[language=java, xleftmargin=0.25\textwidth]
...
if (length = MIN_LENGTH)
    System.out.println("The length is minimal.");
...
  \end{lstlisting}
  %\caption{Fragment1.java.}
  \label{fig:two}
\end{figure}\\
This is of course invalid due to the assignment operator in the \texttt{if}-statement's condition instead of \texttt{==}.
\subsection*{5.9}
We're given the following code segment.
\begin{figure}[h!]
  \centering
  \lstinputlisting[language=java]{%
    /home/brandon/eclipse-workspace/ift_194_hw/src/hw_2/Loop.java}
  \caption{Loop.java.}
  \label{fig:three}
\end{figure}\\
The problem is that it's an infinite loop. Three changes we could make to fix this are
\begin{enumerate}
  \item \texttt{count} could be initialized to a negative integer.
  \item The \texttt{while}-loop's condition could be changed to \texttt{(count < x)}, where $x>50$.
  \item The increment \texttt{count++} could be changed to a decrement \verb|count--|.
\end{enumerate}
\subsection*{5.10}
We're tasked with writing a \texttt{while}-loop that verifies that the user enters a positive integer value. See \autoref{fig:four} for my solution
\begin{figure}[h!]
  \centering
  \lstinputlisting[language=java]{%
    /home/brandon/eclipse-workspace/ift_194_hw/src/hw_2/PositiveInput.java}
  \caption{PositiveInput.java.}
  \label{fig:four}
\end{figure}
\subsection*{PP 5.4}
We're asked to write a simple implementation of the game Hi-Lo. Please see \hyperref[fig:five]{Figure 4} for my solution.
\iftcodefigure{fig:five}{HiLo.java}{/home/brandon/eclipse-workspace/ift_194_hw/src/hw_2/HiLo.java}
\subsection*{6.1}
We're asked to determine how many iterations the following \texttt{for}-loops will execute.
\begin{enumerate}[label=\alph*.]
  \itemsep-0.3em
  \item \texttt{for (int i = 0; i < 20; ++i) \{\}} -- this loop will iterate 20 times.
  \item \texttt{for (int i = 0; i <= 20; ++i) \{\}} -- this loop will iterate 21 times.
  \item \texttt{for (int i = 5; i < 20; ++i) \{\}} -- this loop will iterate 15 times.
  \item \verb|for (int i = 20; i > 0; --i) {}| -- this loop will iterate 20 times.
  \item \texttt{for (int i = 1; i < 20; i += 2) \{\}} -- this loop will iterate 10 times.
  \item \texttt{for (int i = 1; i < 20; i *= 2) \{\}} -- this loop will iterate 5 times.
\end{enumerate}
\subsection*{6.4}
We're asked to transform the following \texttt{while}-loop into an equivalent \texttt{do-while}-loop.
\begin{figure}[h!]
  \centering
  \lstinputlisting[language=java]{%
    /home/brandon/eclipse-workspace/ift_194_hw/src/hw_2/While.java}
  \caption{While.java.}
  \label{fig:five}
\end{figure}\\
I arrive at the following solution.
\begin{figure}[h!]
  \centering
  \lstinputlisting[language=java]{%
    /home/brandon/eclipse-workspace/ift_194_hw/src/hw_2/DoWhile.java}
  \caption{DoWhile.java.}
  \label{fig:six}
\end{figure}
\subsection*{6.8}
We're asked to write a \texttt{for}-loop that will print the multiples of $3$ from $300$ down to $3$. See \autoref{fig:seven} for my solution.
\begin{figure}[h!]
  \centering
  \lstinputlisting[language=java]{%
    /home/brandon/eclipse-workspace/ift_194_hw/src/hw_2/Multiples.java}
  \caption{Multiples.java.}
  \label{fig:seven}
\end{figure}
\subsection*{6.9}
We're asked to write a program that will accept 10 integers from a user and print the largest. See \autoref{fig:eight} for my solution.
\begin{figure}[h!]
  \centering
  \lstinputlisting[language=java]{%
    /home/brandon/eclipse-workspace/ift_194_hw/src/hw_2/Largest.java}
  \caption{Loop.java.}
  \label{fig:eight}
\end{figure}
\subsection*{PP 6.3}
We're asked to write a program that will output a $12\times 12$ multiplication table to the console. Please see \autoref{fig:nine} for my solution.
\begin{figure}[h!]
  \centering
  \lstinputlisting[language=java]{%
    /home/brandon/eclipse-workspace/ift_194_hw/src/hw_2/MultiplicationTable.java}
  \caption{MultiplicationTable.java.}
  \label{fig:nine}
\end{figure}
\section*{Section 2.3}
\subsection*{8.1}
We are tasked with determining which of the following are valid (primitive) array declarations.
\begin{enumerate}[label=\alph*.]
  \item \texttt{int primes = \{ 2, 3, 4, 5, 7, 11 \}} -- this is not a valid array declaration (and 4 is not prime).
  \item \texttt{float elapsedTimes[] = \{ 11.47, 12.04, 11.72, 13.88 \}} -- this is a valid array declaration.
\end{enumerate}
\end{document}
