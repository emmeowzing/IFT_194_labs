\documentclass[leqno, 11pt]{article}

\usepackage{lmodern}
\usepackage[scaled]{beramono}
\usepackage[T1]{fontenc}
\usepackage{amssymb}
\usepackage{amsmath}
\usepackage{hyperref}
\hypersetup{%
  colorlinks=true,
  linkcolor=magenta,
  filecolor=magenta,
  urlcolor=magenta
}

\usepackage[margin=1in]{geometry}
\usepackage{listings}
\usepackage{graphicx}
\usepackage{caption}

\captionsetup{%
  width=1.0\linewidth,
  justification=centering
}

\usepackage{enumitem} % allow characters instead of numbers
\usepackage{verbatimbox}  % center \verbatim env in figure env
\usepackage{siunitx}  % SI unit formatting
\usepackage{colortbl}  % color tabular rows/columns
\usepackage[most]{tcolorbox}

% TODO: Set up captions with tcolorbox so the user doesn't have to explicitly
%       pass a caption={} to each \lstlisting command.
\newtcolorbox[blend into=figures]{codefigure}[1][]{%
  enhanced jigsaw, 
  float=!ht, 
  breakable, 
  frame hidden, 
  colback=white,
  #1
}

% Better `@' symbol
\newcommand{\at}{\mbox{}{\fontfamily{ptm}\selectfont @}\mbox{}}

\newcommand\blfootnote[1]{%
  \begingroup
    \renewcommand\thefootnote{}\footnote{#1}
    \addtocounter{footnote}{-1}
  \endgroup
}

%\graphicspath{"/home/brandon/Desktop/IFT_194/labs/photos"}

\usepackage{xcolor}

\definecolor{tableheaderrow}{HTML}{CED3DB}

\definecolor{javacommentscolor}{HTML}{646464}
\definecolor{javakeywordscolor}{HTML}{7F0055}
\definecolor{javastringscolor}{HTML}{2A00FF}

\definecolor{background}{HTML}{F2F7FF}

\lstset{%
  basicstyle=\linespread{0.8}\scriptsize\ttfamily, % code to be displayed as monospace
  breaklines=true,
  %frame=b
  commentstyle=\color{javacommentscolor},
  keywordstyle=\color{javakeywordscolor},
  stringstyle=\color{javastringscolor},
  showstringspaces=false,  % do not show string spaces character
  tabsize=4,  % change tabs to spaces
  keywordsprefix={@},  % capture method annotations and doctools
  %showtabs=true,
  %tab=|,
  xleftmargin=1cm
}

% TODO: Modify listings in the next lab so that I don't have to wrap in figure
%       env anymore -- makes the following lines irrelevant
\renewcommand{\lstlistingname}{Figure}

\newcommand{\centeredimage}[2]{%
  \begin{center}
    \includegraphics[scale=#1]{#2}
  \end{center}
}

%% Make a multi-page code figure (only way I've figured out how to do this conveniently as of yet).
% Label, caption, and code path
\newcommand{\iftcodefigure}[3]{%
  \begin{codefigure}
    \label{#1}
    \addtocounter{figure}{-1}
    \lstinputlisting[language=java, 
                     backgroundcolor=\color{background},
                     framexleftmargin=10pt,
                     framextopmargin=6pt,
                     framexbottommargin=6pt,
                     frame=tb,
                     framerule=0pt, 
                     caption={#2}, captionpos=b]{#3}
  \end{codefigure}
}

\makeatletter
% Top-align float-only pages
% https://tex.stackexchange.com/a/28565/96382
\setlength{\@fptop}{0pt}
\setlength{\@fpbot}{0pt plus 1fil}

% Set up the table of contents so that titles are displayed
\renewcommand*\contentsname{Summary}
\setcounter{secnumdepth}{0}

% Share Figure and listings counter
\AtBeginDocument{%
  \let\c@figure\c@lstlisting
  \let\thefigure\thelstlisting
  \let\ftype@lstlisting\ftype@figure
}
\makeatother

\title{\vspace{6ex}Object-Oriented Programming in Java\\
  \Large IFT 194: HW 3}
\author{Brandon Doyle\\
\href{mailto:bdoyle@asu.edu}{bdoyle5\at{}asu.edu}\\
1215232174\\[1em]
Dr. Usha Jagannathan\\
\href{mailto:Usha.Jagannathan@asu.edu}{Usha.Jagannathan\at{}asu.edu}}

\setlength{\parindent}{0em}
\setlength{\parskip}{0.5em}

\begin{document}
\begin{titlepage}
\clearpage\maketitle
\thispagestyle{empty}
\end{titlepage}
\tableofcontents
\newpage
\blfootnote{View the source of this document on \href{https://github.com/bjd2385/IFT_194_labs/blob/master/\jobname.tex}{GitHub}.}
\section{Homework 3.1}
\subsection{3.3}
In this question we're asked to declare a \texttt{String} variable and initialize it to contain the same characters as another, except in all uppercase characters.
\begin{lstlisting}[language=java, xleftmargin=0.37\textwidth]
...
String name = "Brandon Doyle";
String str = name.toUpperCase();
...
\end{lstlisting}
\subsection{3.8}
We're tasked with using the \texttt{java.util.Random} class to generate random numbers in the specified ranges. See my solution in \hyperref[fig:one]{Figure 1}.
\iftcodefigure{fig:one}{Rand.java}{%
  /home/brandon/eclipse-workspace/ift_194_hw/src/hw_3/Rand.java}
\subsection{3.9}
In this question we're asked to write an expression that computes the square root of a sum and stores the result in a variable. I would write this as follows.
\begin{lstlisting}[language=java, xleftmargin=0.33\textwidth]
...
double num1 = 1.0, num2 = 2.0;
double num3 = Math.sqrt(num1 + num2);
...
\end{lstlisting}
\subsection{3.10}
In this question we're asked to write a single statement that computes the absolute value of a variable \texttt{total}. I would write this as follows.
\begin{lstlisting}[language=java, xleftmargin=0.35\textwidth]
var absTotal = Math.abs(total);
\end{lstlisting}
Note also that I've used local variable type inference, because there exists an overloaded method for \texttt{long}, \texttt{int}, \texttt{float}, and \texttt{double} types.

Alternatively, we might write something as follows, which does not use any library methods.
\begin{lstlisting}[language=java, xleftmargin=0.28\textwidth]
var absTotal = (total < 0) ? -total : total;
\end{lstlisting}
\subsection{3.11}
In this problem we're asked to write code that will create a \texttt{DecimalFormaat} object to round a value to 4 decimal places. Please see my solution in \hyperref[fig:two]{Figure 2}.
\iftcodefigure{fig:two}{Formatter.java}{%
  /home/brandon/eclipse-workspace/ift_194_hw/src/hw_3/Formatter.java}\\
This program outputs \texttt{3.1416} to my console.

Something I came across that I found rather interesting is there's no need to call \texttt{String.format} in \texttt{System.out.println} because we also have \texttt{System.out.printf}, which is very similar to C's \texttt{printf} function (defined in \texttt{stdio.h}). I'll be sure to remember this while writing future labs and homework.
\subsection{3.12}
In this problem we're tasked with obtaining a double from the user and printing that value to the fourth power to 3 decimal places. Please see \hyperref[fig:three]{Figure 3} for my solution.
\iftcodefigure{fig:three}{GetOutput.java}{%
  /home/brandon/eclipse-workspace/ift_194_hw/src/hw_3/GetOutput.java}\\
The following is an example session.
\begin{verbbox}[\mbox{}\scriptsize]
Enter a double: hello
*** Error: please enter a double
Enter a double: 33.214526252
33.215^4 = 1217060.730
\end{verbbox}
\begin{center}
  \theverbbox
\end{center}
\section{Homework 3.2}
\subsection{4.1}
In this question we're asked to determine which objects in the following pairs may be a subclass and parent class.
\begin{enumerate}[label=\alph*.]
  \itemsep-0.2em
  \item \texttt{Superhero}, \texttt{Superman} -- in this case, \texttt{Superhero} would be the parent class and \texttt{Superman} the subclass.
  \item \texttt{Justin}, \texttt{Person} -- \texttt{Justin} is the subclass, and \texttt{Person} the parent class.
  \item \texttt{Rover}, \texttt{Pet} -- \texttt{Rover} is the subclass and \texttt{Pet} is the superclass.
  \item \texttt{Magazine}, \texttt{Time} -- \texttt{Time} is the subclass and \texttt{Magazine} is the superclass.
  \item \texttt{Christmas}, \texttt{Holiday} -- \texttt{Christmas} is the subclass and \texttt{Holiday} is the subclass.
\end{enumerate}
\subsection{4.4}
Attributes I may include in a class \texttt{Course} to represent a college course include
\begin{itemize}
  \itemsep-0.2em
  \item a list of instructors' names;
  \item the number of credits that can be earned;
  \item the course's name and related attributes, such as the course number;
  \item the start and end dates of the course (could be encoded in \href{https://docs.oracle.com/javase/10/docs/api/java/sql/Date.html}{\texttt{Date}});
  \item location of the class;
  \item and a list of students.
\end{itemize}
Methods or operations I would write include
\begin{itemize}
  \itemsep-0.2em
  \item a method to add or remove students from the course;
  \item a method to modify the location of the course;
  \item and a method to modify the number of credits a course may be.
\end{itemize}
\subsection{4.5}
In this question we're asked to write a method that has no return value and accepts no arguments that prints the lyrics of a song to the console. Please see my solution in \hyperref[fig:four]{Figure 4}.
\iftcodefigure{fig:four}{Lyrics.java}{%
  /home/brandon/eclipse-workspace/ift_194_hw/src/hw_3/Lyrics.java}
\subsection{4.6}
In this problem we're asked to write a method that accepts one integer parameter and returns the value raised to the third power. Please see \hyperref[fig:five]{Figure 5} for my solution.
\iftcodefigure{fig:five}{Cube.java}{%
  /home/brandon/eclipse-workspace/ift_194_hw/src/hw_3/Cube.java}
\subsection{4.10}
In this problem we're asked to write the constructor of a class called \texttt{Movie} that initializes fields \texttt{name} and \texttt{director}. See \hyperref[fig:six]{Figure 6} for my solution.
\iftcodefigure{fig:six}{Movie.java}{%
  /home/brandon/eclipse-workspace/ift_194_hw/src/hw_3/Movie.java}
\subsection{4.11}
In this question we're asked to write a getter and setter method for a field variable \texttt{age} in a class called \texttt{Child}. Please see my solution in \hyperref[fig:seven]{Figure 7}.
\iftcodefigure{fig:seven}{Child.java}{%
  /home/brandon/eclipse-workspace/ift_194_hw/src/hw_3/Child.java}
\end{document}
